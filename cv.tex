\documentclass[10pt,a4]{article}
\topmargin-2.0cm
\advance\oddsidemargin-1.2cm
\advance\evensidemargin-1.2cm
\textheight9.22in
\textwidth6.4in
\newcommand\bb[1]{\mbox{\em #1}}
\def\baselinestretch{1.0}

\usepackage{multicol}
\usepackage{hyperref}

\usepackage{fancyhdr}
\usepackage{origpagecounting}
%\usepackage[dvips]{color}
\usepackage{xcolor}

\newcounter{myEnumCounter}
\newcounter{mySaveCounter}
\renewenvironment{enumerate}{%
  \begin{list}{\arabic{myEnumCounter}.}{\usecounter{myEnumCounter}%
  \setcounter{myEnumCounter}{\value{mySaveCounter}}}
  }{%
  \setcounter{mySaveCounter}{\value{myEnumCounter}}\end{list}%
}
\newcommand\myEnumReset{\setcounter{mySaveCounter}{0}}

\begin{document}

\thispagestyle{fancy}
%\lhead{\textcolor{blue}{\it Vishal Kasliwal}}
\renewcommand{\headrulewidth}{0pt}
\renewcommand{\footrulewidth}{0pt}
\fancyfoot[C]{\footnotesize \textcolor{blue}{}}

\hrule  height 4pt

\vspace*{0.4cm}
\begin{center}
{\huge \textcolor{black}{\bf Vishal Pramod Kasliwal}}\\
%{\bf Ph.D. Candidate in Physics}
%\vspace*{0.25cm}
\end{center}

\begin{small}

\begin{tabbing}
\=xxxxxxxx\=xxxxxxxx\=xxxxxxxx\=\kill
\begin{tabular*}{\linewidth}{l@{\extracolsep{\fill}}r}

Department of Physics \& Astronomy & Phone: 267.206.9287 \\
University of Pennsylvania &  Email: vishal.kasliwal@gmail.com\\
209 S. 33$^{\mathrm{rd}}$ St. & Alt: vish@sas.upenn.edu \\
Philadelphia, PA 19104-6396 & \\
Nationality: Indian
\end{tabular*}
\end{tabbing}

\end{small}

\vspace*{0.1cm}

\subsection*{EMPLOYMENT}
\hrule  height 4pt
\vspace{0.2cm}

\begin{itemize}

\item Postdoctoral Fellow in LSST Data Management \& Galaxy Surveys ({\it Sept. 2015 - present}) \\
\begin{tabbing}
xxxx\=xxxxxxxx\=xxxxxxxx\=xxxxxxxx\=\kill
\>\begin{tabular*}{0.9\linewidth}{l@{\extracolsep{\fill}}r}
Univ. of Pennsylvania, Dept. of Physics \& Astronomy & \\
Princeton Univ., Dept. of Astrophysical Sciences & \\
Supervisors: Dr. Robert Lupton, Dr. Bhuvnesh Jain, \& Dr. Mike Jarvis \\
 & \\
\end{tabular*}
\end{tabbing}

\end{itemize}

\subsection*{EDUCATION}
\hrule  height 4pt
\vspace{0.2cm}

\begin{tabbing}
xxxx\=xxxxxxxx\=xxxxxxxx\=xxxxxxxx\=\kill
\>\begin{tabular*}{0.9\linewidth}{l@{\extracolsep{\fill}}r}

Drexel University & {\it September 2015} \\
Ph.D. in Physics  & \\
Thesis: \it{Probing AGN Accretion Physics through AGN Variability:} \\
\hspace{1.1cm} \it{Insights from Kepler} \\
Advisors: Dr. Michael S. Vogeley \& Dr. Gordon T. Richards \\
 & \\

Virginia Commonwealth University & {\it May 2007} \\
M.S. in Physics \& Applied Physics & \\
Thesis: \it{CAFM Studies of Epitaxial} \\
\hspace{1.1cm} \it{Lateral Overgrowth GaN Films} \\
Advisor: Dr. Alison A. Baski \\
 & \\

University of Richmond & {\it May 2005} \\
B.A. in Mathematics \& Physics  & \\
Thesis: \it{The Bispectrum as a Quantifier of non-Gaussianity} \\
\hspace{1.1cm} \it{in the Cosmic Microwave 	Background} \\
Advisor: Dr. Emory F. Bunn \\

\end{tabular*}
\end{tabbing}

\subsection*{PUBLICATIONS}
\hrule  height 4pt
\vspace{0.2cm}

\begin{itemize}
	\item ``Science-driven Optimization of the LSST Observing Strategy", 2016 (in prep.)
	\item ``Extracting Information from AGN Variability", 2016 (\textit{submitted to MNRAS})
	\item ``The LSST Data Management System", \href{http://adsabs.harvard.edu/cgi-bin/bib_query?arXiv:1512.07914}{Proceedings of ADASS XXV}, 2015
	\item ``Do the Kepler AGN light curves need reprocessing?", \href{http://dx.doi.org/10.1093/mnras/stv1797}{MNRAS, 453, 2075}, 2015
	\item ``Are the variability properties of the Kepler AGN light curves consistent with a damped random walk?", \href{http://dx.doi.org/ 10.1093/mnras/stv1230}{MNRAS, 451, 4328}, 2015
	\item  ``Thirty Meter Telescope Detailed Science Case: 2015", \href{http://arxiv.org/abs/1505.01195}{http://arxiv.org/abs/1505.01195}, 2015
	\item  ``AFM and CAFM studies of ELO GaN films", \href{http://dx.doi.org/10.1117/12.706773}{Proc. SPIE 6473, 647308}, 2007
	\item ``Local electronic and optical behaviors of a-plane GaN grown via epitaxial lateral overgrowrth", \href{http://dx.doi.org/10.1063/1.2429901}{Appl. Phys. Lett., 90, 011913}, 2007
\end{itemize}

\myEnumReset

\subsection*{MEMBERSHIP IN PROFESSIONAL ORGANIZATIONS}
\hrule  height 4pt
\vspace{0.2cm}

\begin{itemize}
	\item American Astronomical Society (AAS)
	\item Large Synoptic Survey Telescope (LSST) Data Management (DM)
	\item Thirty Meter Telescope (TMT) International Science Development Team (ISDT): Time Domain Science
	\item Thirty Meter Telescope (TMT) International Science Development Team (ISDT): Supermassive Black Holes
  \item LSST Galaxies Collaboration
	\item $\Sigma \Pi \Sigma$ Drexel University, Philadelphia, PA.
	\item $\Sigma \Pi \Sigma$ Virginia Commonwealth University, Richmond, VA.
\end{itemize}

\myEnumReset

\subsection*{CONFERENCE \& MEETING PARTICIPATION}
\hrule  height 4pt
\vspace{0.2cm}

\begin{itemize}

    \item Participated in the LSST 2016 Project \& Community Workshop, August 2016, Tuscon, AZ

    \item Presented {\it Probing Accretion Processes through Variability} at the 2016 TMT Science Forum `International Partnership for Global Astronomy', May 2016, Kyoto, Japan.

    \item Presented {\it AGN Variability: Insights from Kepler} in the Princeton HSC Science Discussion Series, March 2016, Princeton, NJ.

    \item Presented {\it AGN Variability on Short Timescales: What does Kepler tell us about AGN Variability?} at the 2015 TMT Science Forum `Maximizing Transformative Science with TMT', June 2015, Washington, DC.

    \item Presented {\it What can Kepler tell us about AGN variability?} at the 225th Meeting of the American Astronomical Society, January 2015, Seattle, WA.

    \item Presented {\it Do Kepler AGN Light Curves Exhibit a Damped Random Walk?} at the 224th Meeting of the American Astronomical Society, June 2014, Boston, MA.

    \item Participated in the SciCoder Workshop, June 2010, New York, NY

    \item Attended the 215th Meeting of the American Astronomical Society, Jan. 2010, Washington, DC.

    \item Participated in the NSF-PIRE Summer School: Lensing of the CMB and High-z Galaxies, July. 2009, Philadelphia, PA.

    \item  Presented {\it The Bispectrum of Galactic Dust: Implications for Microwave Background non-Gaussianity} at the 204th Meeting of the American Astronomical Society, May 2004, Denver, CO.

\end{itemize}

\myEnumReset

\subsection*{COMPUTING}
\hrule  height 4pt
\vspace{0.2cm}
\begin{itemize}


\item Proficient in using C++, Python \& Cython for
	\begin{enumerate}
		\item Scientific computing, data visualization, \& numerical optimization.
		\item Parallel computing with OpenMP, Intel Cilk Plus, and the Python Multiprocessing module.
		\item Programming Intel Xeon Phi accelerator cards using Intel LEO extensions \& OpenMp 4.0 in C++.
		\item Hardware random number generation using Intel Bull Mountain technology.
	\end{enumerate}
\item 1 year of experience developing LSST Stack software in a collaborative professional environment with regular usage of standard development tools and techniques for agile development, continuous integration, and version control. Tools used include Atlassian JIRA, Jenkins, and Git.
\item Principle developer of C++, Python, \& Cython library \textsc{k\={a}l\={i}} for light-curve analysis using stochastic models including Continuous-time Autoregressive-Moving Average processes.
\item 12+ years of experience with Linux, \LaTeX, Mathematica, and MS Windows.
\item 7 years of experience using Mac OS X for programming and development.
\item Experience with IDL, bash, SQL, R, Intel CompilerXE toolchain, gcc toolchain, MATLAB, LONCAPA, Photoshop and Office Suites including MS Office, OpenOffice \& LibreOffice.

\end{itemize}

\myEnumReset

\subsection*{PREVIOUS EMPLOYMENT}
\hrule  height 4pt
\vspace{0.2cm}

\begin{itemize}

\item Graduate Research Assistant ({\it April 2014 - Sept. 2015}) \\
\begin{tabbing}
xxxx\=xxxxxxxx\=xxxxxxxx\=xxxxxxxx\=\kill
\>\begin{tabular*}{0.9\linewidth}{l@{\extracolsep{\fill}}r}
Drexel Univ., Dept. of Physics & \\
Advisors: Dr. Michael S. Vogeley \& Dr. Gordon T. Richards \\
 & \\
\end{tabular*}
\end{tabbing}

\item Graduate Teaching Assistant ({\it Sept. 2008 - March 2014}) \\
\begin{tabbing}
xxxx\=xxxxxxxx\=xxxxxxxx\=xxxxxxxx\=\kill
\>\begin{tabular*}{0.9\linewidth}{l@{\extracolsep{\fill}}r}
Drexel Univ., Dept. of Physics & \\
Supervisors: Dr. Michel Vallieres & \\
Courses Taught: {\it Quantum Mechanics I, II, \& III} & \\ % PHYS 326, 327, \& 428
\ \ \ \ \ \ \ \ \ \ \ \ \ \ \ \ \ \ \ \ \ \ \ {\it Fundamentals of Physics I \& II} & \\ % PHYS 101, 102
\ \ \ \ \ \ \ \ \ \ \ \ \ \ \ \ \ \ \ \ \ \ \ {\it Introductory Physics I} & \\ % PHYS 152
 & \\
\end{tabular*}
\end{tabbing}

\item Adjunct Instructor ({\it June 2007 - June 2008}) \\
\begin{tabbing}
xxxx\=xxxxxxxx\=xxxxxxxx\=xxxxxxxx\=\kill
\>\begin{tabular*}{0.9\linewidth}{l@{\extracolsep{\fill}}r}
Virginia Commonwealth Univ., Dept. of Physics & \\
Supervisor: Dr. Alison A. Baski & \\
Courses Taught: {\it Elementary Astronomy} & \\ % PHYS 103
\ \ \ \ \ \ \ \ \ \ \ \ \ \ \ \ \ \ \ \ \ \ \ {\it General Physics I \& II} & \\ % PHYS 201, 202
\ \ \ \ \ \ \ \ \ \ \ \ \ \ \ \ \ \ \ \ \ \ \ {\it University Physics I \& II} & \\ % PHYS 207, 208
\ \ \ \ \ \ \ \ \ \ \ \ \ \ \ \ \ \ \ \ \ \ \ {\it Guided Inquiry for University Physics I \& II} & \\ % PHYS 351, 352
 & \\
\end{tabular*}
\end{tabbing}

\item Graduate Teaching Assistant ({\it Aug. 2005 - May 2007}) \\
\begin{tabbing}
xxxx\=xxxxxxxx\=xxxxxxxx\=xxxxxxxx\=\kill
\>\begin{tabular*}{0.9\linewidth}{l@{\extracolsep{\fill}}r}
Virginia Commonwealth Univ., Dept. of Physics & \\
Supervisor: Dr. Alison A. Baski & \\
Courses Taught: {\it Elementary Astronomy} & \\ % PHYS 103
\ \ \ \ \ \ \ \ \ \ \ \ \ \ \ \ \ \ \ \ \ \ \ {\it General Physics I \& II} & \\ % PHYS 201, 202
\ \ \ \ \ \ \ \ \ \ \ \ \ \ \ \ \ \ \ \ \ \ \ {\it University Physics I \& II} & \\ % PHYS 207, 208
 & \\
\end{tabular*}
\end{tabbing}

\item Graduate Research Assistant ({\it Summer 2006}) \\
\begin{tabbing}
xxxx\=xxxxxxxx\=xxxxxxxx\=xxxxxxxx\=\kill
\>\begin{tabular*}{0.9\linewidth}{l@{\extracolsep{\fill}}r}
Virginia Commonwealth Univ., Dept. of Physics & \\
Advisor: Dr. Alison A. Baski \\
 & \\
\end{tabular*}
\end{tabbing}

\item Research Assistant ({\it May 2003 - May 2005}) \\
\begin{tabbing}
xxxx\=xxxxxxxx\=xxxxxxxx\=xxxxxxxx\=\kill
\>\begin{tabular*}{0.9\linewidth}{l@{\extracolsep{\fill}}r}
Univ. of Richmond, Dept. of Physics & \\
Advisor: Dr. Emory F. Bunn \\
 & \\
\end{tabular*}
\end{tabbing}

\item Computing Lab Assistant ({\it Jan. 2002 - May 2005}) \\
\begin{tabbing}
xxxx\=xxxxxxxx\=xxxxxxxx\=xxxxxxxx\=\kill
\>\begin{tabular*}{0.9\linewidth}{l@{\extracolsep{\fill}}r}
Univ. of Richmond, Information Services & \\
Supervisor: Vicki F. Brady \\
 & \\
\end{tabular*}
\end{tabbing}

\end{itemize}

%\subsection*{RESEARCH INTERESTS}
%\hrule  height 4pt
%\vspace{0.2cm}
%\begin{list}{}{}
%\item I am interested in studying accretion processes around compact objects. I model the luminosity fluctuations observed in accreting systems using time series methods to probe the physical processes occuring in the flow.
%\end{list}

\subsection*{ACADEMIC HONORS}
\hrule  height 4pt
\vspace{0.2cm}
\begin{itemize}

	\item Jackson J. Taylor Best Senior Seminar in Physics Award, University of Richmond, 2005.

        \item Marsh White Award for the Outstanding Undergraduate Paper at the Society of Physics Students Undergraduate Research Session, Southeastern Section of the American Physical Society, 2003.

	\item National level participant in the Mathematics Training and Talent Search Programme (I.I.T., Mumbai), 2002.

	\item National level participant in the 2nd Indian Astronomy Olympiad, I.S.R.O., 2000.

\end{itemize}

\subsection*{SERVICE AND OUTREACH}
\hrule  height 4pt
\vspace{0.2cm}
\begin{itemize}
	\item Started the \textit{The Sky in the City} astronomy night program
	at the Dornsife Center (Drexel University). Responsiblities include
	running the program for the Drexel Physics Department and acquisition
	of telescopes to support the event. \\
	May 2015 - {\it present}

	\item Volunteer at the Drexel University Lynch Observatory for
          telescope open houses. Responsibilities include setting up,
          operating, and storage of the department's telescopes. \\
          Sept. 2008 - {\it present}

	\item Organized and co-taught the ``Fun Physics" lectures at
	Drexel University Department of Physics. Topics included General
	Relativity, Advanced Mathematical Physics, \& Spinor Physics. \\
	Fall 2008 - Fall 2009
\end{itemize}

%\subsection*{PRESENTATIONS}
%\hrule  height 4pt
%\vspace{0.2cm}

%\subsubsection*{TALKS}
%\begin{enumerate}

    %\item {\it Variability as a Probe of the Physics of Active Galactic Nuclei}, Drexel University College of Arts \& Sciences Research Day, Philadelphia, PA, April 2013.

    %\item {\it How do Active Galactic Nuclei Vary in Brightness?}, Drexel University Books \& Bagels Seminar, Philadelphia, PA, Jan. 2013.

    %\item {\it The Variability of Active Galactic Nuclei}, Drexel University Books \& Bagels Seminar, Philadelphia, PA, May 2012.

    %\item {\it Probing the nature of Dark Energy and finding Large Scale Structure with the LSST}, Drexel University PhD Qualifier Oral Presentation, Philadelphia, PA, May 2009.

    %\item {\it CAFM Studies of Epitaxial Lateral Overgrowth GaN}, Virginia Commonwealth University Master's Thesis Presentation, Richmond, VA, May 2007.

    %\item {\it The Bispectrum as a Quantifier of non-Gaussianity in the CMB}, University of Richmond Undergraduate Thesis Presentation, Richmond, VA, May 2005.

%\end{enumerate}

%\myEnumReset

%\subsubsection*{POSTERS}

%\begin{enumerate}

%\item {\it Do Kepler AGN Light Curves Exhibit a Damped Random Walk?}, The
%  224th Meeting of the American Astronomical Society, Jun 2014,
%  Boston, MA.

%\item {\it The Bispectrum of Galactic Dust: Implications for Microwave Background non-Gaussianity}, The
%  204th Meeting of the American Astronomical Society, May 2004,
%  Denver, CO.

%\item Various posters for both Drexel University Research Day, and The
% College of Arts and Sciences Research Day, 2009-2014, Drexel University,
% Philadelphia, PA.

%\end{enumerate}

\newpage
\pagestyle{fancy}
\lhead{\textcolor{black}{\it Vishal Kasliwal}}
\fancyfoot[C]{}

\subsection*{REFERENCES}
\hrule  height 4pt
\vspace{0.2cm}

\begin{itemize}
	\item Dr. Michael S. Vogeley \\
	Director of Graduate Studies; Professor
	\begin{tabbing}
	\=xxxxxxxx\=xxxxxxxx\=xxxxxxxx\=\kill
	\begin{tabular*}{\linewidth}{l@{\extracolsep{\fill}}r}

	Dept. of Physics & Phone: (215)895-2710 \\
	Drexel Univ. &  Email: vogeley@drexel.edu \\
	3141 Chestnut Street & \\
	Philadelphia, PA 19104 & \\
	\end{tabular*}
	\end{tabbing}

	\item Dr. Gordon T. Richards \\
        	Associate Professor
	\begin{tabbing}
	\=xxxxxxxx\=xxxxxxxx\=xxxxxxxx\=\kill
	\begin{tabular*}{\linewidth}{l@{\extracolsep{\fill}}r}

	Dept. of Physics & Phone: (215)895-2713 \\
	Drexel Univ. &  Email: gtr@physics.drexel.edu \\
	3141 Chestnut Street & \\
	Philadelphia, PA 19104 & \\
	\end{tabular*}
	\end{tabbing}

        \item Dr. Stephen L.W. McMillan \\
        Interim Department Head; Professor
	\begin{tabbing}
	\=xxxxxxxx\=xxxxxxxx\=xxxxxxxx\=\kill
	\begin{tabular*}{\linewidth}{l@{\extracolsep{\fill}}r}

	Dept. of Physics & Phone: (215)895-2709 \\
	Drexel Univ. &  Email: steve@physics.drexel.edu \\
	3141 Chestnut Street & \\
	Philadelphia, PA 19104 & \\
	\end{tabular*}
	\end{tabbing}

        \item Dr. Robert Gilmore \\
        	Professor
	\begin{tabbing}
	\=xxxxxxxx\=xxxxxxxx\=xxxxxxxx\=\kill
	\begin{tabular*}{\linewidth}{l@{\extracolsep{\fill}}r}

	Dept. of Physics & Phone: (215)895-2779 \\
	Drexel Univ. &  Email: robert.gilmore@drexel.edu \\
	3141 Chestnut Street & \\
	Philadelphia, PA 19104 & \\
	\end{tabular*}
	\end{tabbing}

\end{itemize}

\vspace{0.1cm}

\end{document}

