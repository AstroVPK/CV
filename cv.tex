\documentclass[10pt,a4]{article}
\topmargin-2.0cm
\advance\oddsidemargin-1.2cm
\advance\evensidemargin-1.2cm
\textheight9.22in
\textwidth6.4in
\newcommand\bb[1]{\mbox{\em #1}}
\def\baselinestretch{1.0}

\usepackage{multicol}
\usepackage{hyperref}
\usepackage{verbatim}
\usepackage{fancyhdr}
\usepackage{origpagecounting}
%\usepackage[dvips]{color}
\usepackage{xcolor}

\newcounter{myEnumCounter}
\newcounter{mySaveCounter}
\renewenvironment{enumerate}{%
  \begin{list}{\arabic{myEnumCounter}.}{\usecounter{myEnumCounter}%
  \setcounter{myEnumCounter}{\value{mySaveCounter}}}
  }{%
  \setcounter{mySaveCounter}{\value{myEnumCounter}}\end{list}%
}
\newcommand\myEnumReset{\setcounter{mySaveCounter}{0}}

\begin{document}

\thispagestyle{fancy}
%\lhead{\textcolor{blue}{\it Vishal Kasliwal}}
\renewcommand{\headrulewidth}{0pt}
\renewcommand{\footrulewidth}{0pt}
\fancyfoot[C]{\footnotesize \textcolor{blue}{}}

\hrule  height 4pt

\vspace*{0.4cm}
\begin{center}
{\huge \textcolor{black}{\bf Vishal Pramod Kasliwal}}\\
%{\bf Ph.D. Candidate in Physics}
%\vspace*{0.25cm}
\end{center}

\begin{small}

\begin{tabbing}
\=xxxxxxxx\=xxxxxxxx\=xxxxxxxx\=\kill
\begin{tabular*}{\linewidth}{l@{\extracolsep{\fill}}r}

Wave Computing & Phone: 267.206.9287 \\
42 W Campbell Ave., Suite 301 & Email: vishal.kasliwal@gmail.com \\
Campbell, CA 95008 & Alt: vishal@wavecomp.com \\
United States of America & \\
& Nationality: Indian\\
& Languages Spoken: English \& Hindi\\
\end{tabular*}
\end{tabbing}

\end{small}

\vspace*{0.1cm}

\subsection*{EMPLOYMENT}
\hrule  height 4pt
\vspace{0.2cm}
\begin{enumerate}

\item Research \& Development Software Engineer at Wave Computing ({\it December 2017 - present}) \\

Wave Computing is developing the next-generation of solutions for speeding up Deep Learning applications using Dataflow Processing Units (DPUs), which contain thousands of interconnected dataflow Processing Elements (PEs). DPUs power Wave Computing's custom appliance for developing, testing, and deploying Deep Learning models. I develop agents (kernel and data movement software kernels) which are executed by Wave Computing's Dataflow Computers for Deep Learning acceleration.. Kernels implemented include
\begin{itemize}
    \item IEEE 754-2008 compliant rounding kernels.
    \item forward \& backward propagation for various activation functions including ReLU.
    \item backward propagation for bias.
\end{itemize}

\item High Performance Computing (HPC) Research Engineer at Colfax International ({\it March 2017 - December 2017}) \\

I conducted research on new HPC technologies and provide HPC consulting services to clients across a broad range of industries.

\begin{itemize}
    \item I parallelized an oil \& gas sector client's Computational Fluid Dynamics (CFD) reservoir simulation leading to a $>$ 10X speed-up. I re-factored the 40,000$+$ line C codebase to enable domain-decomposition for further performance improvement.
    \item I performed a detailed analysis of the performance potential of various commercial C++-compilers by
      \begin{itemize}
        \item developing a test-suite of scientific computational kernels
        \item quantifying the performance of the code generated by each compiler for each test in the suite
        \item analyzing the assembly instructions generated by each compiler in order to understand the perforamance behavior
      \end{itemize}
      The technical whitepaper describing my findings may be found at \href{https://colfaxresearch.com/compiler-comparison/}{Colfax Research}.
    \item I presented a lecture on \href{https://software.intel.com/en-us/advisor}{Intel Advisor} for Stanford University's ME344: Introduction to High Performance Computing course on July, 20th, 2017.
\end{itemize}

\item Postdoctoral Fellow in LSST Data Management \& Galaxy Surveys ({\it Sept. 2015 - Feb. 2017}) \\
\begin{tabbing}
xxxx\=xxxxxxxx\=xxxxxxxx\=xxxxxxxx\=\kill
\>\begin{tabular*}{0.9\linewidth}{l@{\extracolsep{\fill}}r}
Univ. of Pennsylvania, Dept. of Physics \& Astronomy & \\
Princeton Univ., Dept. of Astrophysical Sciences & \\
Supervisors: Dr. Bhuvnesh Jain, Dr. Robert Lupton & \\
\ \ \ \ \ \ \ \ \ \ \ \ \ \ \ \ \ Dr. Adam Lidz, Dr John Swinbank, & \\
\ \ \ \ \ \ \ \ \ \ \ \ \ \ \ \ \ \& Dr. Mike Jarvis & \\
\end{tabular*}
\end{tabbing}

My postdoctoral duties were split between my responsibilities as an algorithms and software developer on the software
stack for the \href{https://www.lsst.org/}{Large Synoptic Survey Telescope (LSST)} with the Princeton LSST Data Management
group and my research on the analysis of stochastic light curves from accretion flows around binary supermassive
black-holes at UPenn.

Work done at Princeton:
\begin{itemize}
    \item optimal image stacking
    \item covariance propagation through image co-addition
    \item star galaxy separation using machine learning techniques
    \item functionality and bug-fix contributions to the LSST software stack
\end{itemize}

Work done at UPenn:
\begin{itemize}
    \item developed an algorithm to identify potential binary supermassive black hole candidates from time-domain data.
    \item implemented a new module in the light-curve software library - \href{https://github.com/AstroVPK/kali}{\textsc{k\={a}l\={i}}}
    to determine the orbital parameters of supermassive binaries from time-domain data using Monte-Carlo Markov Chain methods.
\end{itemize}

\end{enumerate}
\myEnumReset

\subsection*{EDUCATION}
\hrule  height 4pt
\vspace{0.2cm}

\begin{tabbing}
xxxx\=xxxxxxxx\=xxxxxxxx\=xxxxxxxx\=\kill
\>\begin{tabular*}{0.9\linewidth}{l@{\extracolsep{\fill}}r}

Drexel University & {\it September 2015} \\
Ph.D. in Physics  & \\
Thesis: \it{Probing AGN Accretion Physics through AGN Variability:} \\
\hspace{1.1cm} \it{Insights from Kepler} \\
Advisors: Dr. Michael S. Vogeley \& Dr. Gordon T. Richards \\
Students Supervised: Jackie Moreno (Graduate), Jack O'Brien (Undergraduate), \\
\hspace{3.3cm} \& Brandon Rupert (Undergraduate) \\
 & \\

Virginia Commonwealth University & {\it May 2007} \\
M.S. in Physics \& Applied Physics & \\
Thesis: \it{CAFM Studies of Epitaxial} \\
\hspace{1.1cm} \it{Lateral Overgrowth GaN Films} \\
Advisor: Dr. Alison A. Baski \\
 & \\

University of Richmond & {\it May 2005} \\
B.A. in Mathematics \& Physics  & \\
Thesis: \it{The Bispectrum as a Quantifier of non-Gaussianity} \\
\hspace{1.1cm} \it{in the Cosmic Microwave 	Background} \\
Advisor: Dr. Emory F. Bunn \\

\end{tabular*}
\end{tabbing}

\subsection*{STATISTICS, ANALYTICS \& COMPUTING}
\hrule  height 4pt
\vspace{0.2cm}
\begin{itemize}
\item Proficient in C, C++, Python \& Cython for
	\begin{enumerate}
		\item scientific computing, statistics \& statistical modeling, pattern recognition, data science, data analysis, data modeling, analytics, machine learning.
		\item code optimization including parallel computing with OpenMP, Intel Cilk Plus, and the Python Multiprocessing module.
		\item programming Intel Xeon Phi Knight's Landing CPUs and Intel Xeon Phi Knight's Corner accelerator cards using Intel LEO extensions \& OpenMP 4.5 in C \& C++.
		\item generating hardware random numbers using Intel Bull Mountain technology.
    \item creating frameworks for scientific analysis using design patterns \& object-oriented design principles.
    \item algorithm development with Intel Math Kernel Library, Intel Data Analytics Library, scipy, numpy, \& scikit-learn.
    \item unit-testing with py.test and C++ Boost UTF.
	\end{enumerate}
\item Proficient in programming Deep Learning computational kernels in WFG, Wave Computing's proprietary programming language for programming Data Flow Processing Units (DPUs).
\item 10 months experience speeding-up oil \& gas sector client CFD codebase \& refactoring codebase to permit domain-decomposition.
\item 1.5 year of experience developing LSST Stack software in a collaborative professional environment with regular usage of standard development tools and techniques for agile development, continuous integration, and version control. Tools used include Atlassian JIRA, Jenkins, and Git.
\item Principle developer of C++, Python, \& Cython library \href{https://github.com/AstroVPK/kali}{\textsc{k\={a}l\={i}}} for light-curve analysis using stochastic models including Continuous-time Autoregressive-Moving Average (C-ARMA) \& modulated C-ARMA processes.
\item 12+ years of experience with Linux, \LaTeX, Mathematica, and MS Windows \& 7 years of experience with Mac OS X for programming and development.
\item Experience with IDL, bash, SQL, R, Intel CompilerXE toolchain, gcc toolchain, MATLAB, LONCAPA, Photoshop and Office Suites including MS Office, OpenOffice \& LibreOffice.
\end{itemize}
\myEnumReset

\subsection*{PUBLICATIONS}
\hrule  height 4pt
\vspace{0.2cm}
\begin{itemize}
    \item ``A Performance-Based Comparison of C/C++ Compilers", \href{https://colfaxresearch.com/compiler-comparison/}{https://colfaxresearch.com/compiler-comparison/}, 2017
	\item  ``Large Synoptic Survey Telescope Galaxies Science Roadmap", \href{https://arxiv.org/abs/1708.01617}{https://arxiv.org/abs/1708.01617}, 2017
	\item $\mathbf{^{*}}$``Extracting Information from AGN Variability", \href{https://doi.org/10.1093/mnras/stx1420}{MNRAS, 470, 3, 3027-3048}, 2017
	\item ``Science-driven Optimization of the LSST Observing Strategy", \href{https://github.com/LSSTScienceCollaborations/ObservingStrategy}{GitHub Repository}, 2016
	\item ``The LSST Data Management System", \href{http://adsabs.harvard.edu/cgi-bin/bib_query?arXiv:1512.07914}{Proceedings of ADASS XXV}, 2015
	\item $\mathbf{^{*}}$``Do the Kepler AGN light curves need reprocessing?", \href{http://dx.doi.org/10.1093/mnras/stv1797}{MNRAS, 453, 2075}, 2015
	\item $\mathbf{^{*}}$``Are the variability properties of the Kepler AGN light curves consistent with a damped random walk?", \href{http://dx.doi.org/ 10.1093/mnras/stv1230}{MNRAS, 451, 4328}, 2015
	\item  ``Thirty Meter Telescope Detailed Science Case: 2015", \href{http://arxiv.org/abs/1505.01195}{http://arxiv.org/abs/1505.01195}, 2015
	\item  ``AFM and CAFM studies of ELO GaN films", \href{http://dx.doi.org/10.1117/12.706773}{Proc. SPIE 6473, 647308}, 2007
	\item ``Local electronic and optical behaviors of a-plane GaN grown via epitaxial lateral overgrowrth", \href{http://dx.doi.org/10.1063/1.2429901}{Appl. Phys. Lett., 90, 011913}, 2007
\end{itemize}
\myEnumReset

\subsection*{CONFERENCE \& MEETING PARTICIPATION}
\hrule  height 4pt
\vspace{0.2cm}
\begin{itemize}
    \item Presented a lecture on Intel Advisor at Stanford University for the class ME344: Introduction to High Performance Computing, July 20th, 2017, Stanford, CA
    \item Presented {\it Optical Variability Signatures from Massive Black Hole Binaries} at the 229$^{\mathrm{th}}$ Meeting of the American Astronomical Society, January 7$^{\mathrm{th}}$, 2017, Grapevine, TX
    \item Presented {\it Extracting Information From AGN Variability: an LSST AGN Collaboration Proposal} at the 2017 LSST AGN Science Collaboration Roadmap Development Meeting, January 3$^{\mathrm{rd}}$, 2017, Grapevine, TX
    \item Presented {\it Extracting Information from AGN Variability} at the 2016 KARL LSST Workshop, November 2016, Louisville, KY
    \item Presented {\it Surveying the Dynamic Sky with the LSST} at the 2016 KARL LSST Workshop, November 2016, Louisville, KY
    \item Presented {\it AGN Variability: Insights from Kepler} at the 2016 Hotwiring the Transient Universe V Meeting, October 2016, Villanova, PA
    \item Participated in the LSST 2016 Project \& Community Workshop, August 2016, Tuscon, AZ
    \item Presented {\it Probing Accretion Processes through Variability} at the 2016 TMT Science Forum `International Partnership for Global Astronomy', May 2016, Kyoto, Japan.
    \item Presented {\it AGN Variability: Insights from Kepler} in the Princeton HSC Science Discussion Series, March 2016, Princeton, NJ.
    \item Presented {\it AGN Variability on Short Timescales: What does Kepler tell us about AGN Variability?} at the 2015 TMT Science Forum `Maximizing Transformative Science with TMT', June 2015, Washington, DC.
    \item Presented {\it What can Kepler tell us about AGN variability?} at the 225th Meeting of the American Astronomical Society, January 2015, Seattle, WA.
    \item Presented {\it Do Kepler AGN Light Curves Exhibit a Damped Random Walk?} at the 224th Meeting of the American Astronomical Society, June 2014, Boston, MA.
    \item Participated in the SciCoder Workshop, June 2010, New York, NY
    \item Attended the 215th Meeting of the American Astronomical Society, Jan. 2010, Washington, DC.
    \item Participated in the NSF-PIRE Summer School: Lensing of the CMB and High-z Galaxies, July. 2009, Philadelphia, PA.
    \item  Presented {\it The Bispectrum of Galactic Dust: Implications for Microwave Background non-Gaussianity} at the 204th Meeting of the American Astronomical Society, May 2004, Denver, CO.
\end{itemize}
\myEnumReset

\subsection*{GRANTS, OBSERVING PROPOSALS, SERVICE, \& AWARDS}
\hrule  height 4pt
\vspace{0.2cm}
\begin{itemize}
  \item Peer reviewer for {\it The Astrophysical Journal}.
  \item Peer reviewer for {\it Monthly Notices of the Royal Astronomical Society}.
  \item Co-investigator on K2 GO16088, K2 GO14088, K2 GO12013, K2 GO8052, \& K2 GO10052 Observing Campaigns, University of Pennsylvania, 2016
    \begin{itemize}
      \item Dr. Gordon T. Richards (PI)
      \item Dr Michael S. Vogeley (CoI)
    \end{itemize}
  \item Helped write NASA grant NNX14AL56G, Drexel University, 2014 - 2017.
    \begin{itemize}
      \item Dr. Gordon T. Richards (PI)
      \item Dr Michael S. Vogeley (CoI)
    \end{itemize}
	\item Jackson J. Taylor Best Senior Seminar in Physics Award, University of Richmond, 2005.
  \item Marsh White Award for the Outstanding Undergraduate Paper at the Society of Physics Students Undergraduate Research Session, Southeastern Section of the American Physical Society, 2003.
	\item National level participant in the Mathematics Training and Talent Search Programme (I.I.T., Mumbai), 2002.
	\item National level participant in the 2nd Indian Astronomy Olympiad, I.S.R.O., 2000.
\end{itemize}
\myEnumReset

\subsection*{MEMBERSHIP IN PROFESSIONAL ORGANIZATIONS}
\hrule  height 4pt
\vspace{0.2cm}
\begin{itemize}
	\item Full Member of the American Astronomical Society (AAS)
	\item Member of the Large Synoptic Survey Telescope (LSST) Data Management (DM)
	\item Member of the Thirty Meter Telescope (TMT) International Science Development Team (ISDT): Time Domain Science
	\item Member of the Thirty Meter Telescope (TMT) International Science Development Team (ISDT): Supermassive Black Holes
    \item Member of the LSST Galaxies Collaboration
	\item $\Sigma \Pi \Sigma$ Drexel University, Philadelphia, PA.
	\item $\Sigma \Pi \Sigma$ Virginia Commonwealth University, Richmond, VA.
\end{itemize}
\myEnumReset

\subsection*{SERVICE \& OUTREACH}
\hrule  height 4pt
\vspace{0.2cm}
\begin{itemize}
	\item Started the \textit{The Sky in the City} astronomy night program
	at the Dornsife Center (Drexel University). Responsiblities include
	running the program for the Drexel Physics Department and acquisition
	of telescopes to support the event. \\
	May 2015 - {\it present}
	\item Volunteer at the Drexel University Lynch Observatory for
          telescope open houses. Responsibilities include setting up,
          operating, and storage of the department's telescopes. \\
          Sept. 2008 - {\it present}
	\item Organized and co-taught the ``Fun Physics" lectures at
	Drexel University Department of Physics. Topics included General
	Relativity, Advanced Mathematical Physics, \& Spinor Physics. \\
	Fall 2008 - Fall 2009
\end{itemize}
\myEnumReset

\subsection*{PRIOR POSITIONS HELD}
\hrule  height 4pt
\vspace{0.2cm}


%\begin{comment}


\begin{itemize}
    \item Graduate Research Assistant ({\it April 2014 - Sept. 2015}) \\
    \item Graduate Teaching Assistant ({\it Sept. 2008 - March 2014}) \\
    \item Adjunct Instructor ({\it June 2007 - June 2008}) \\
    \item Graduate Teaching Assistant ({\it Aug. 2005 - May 2007}) \\
    \item Research Assistant ({\it May 2003 - May 2005}) \\
    \item Computing Lab Assistant ({\it Jan. 2002 - May 2005}) \\
\end{itemize}

%\end{comment}




\begin{comment}


\begin{itemize}

\item Graduate Research Assistant ({\it April 2014 - Sept. 2015}) \\
\begin{tabbing}
xxxx\=xxxxxxxx\=xxxxxxxx\=xxxxxxxx\=\kill
\>\begin{tabular*}{0.9\linewidth}{l@{\extracolsep{\fill}}r}
Drexel Univ., Dept. of Physics & \\
Advisors: Dr. Michael S. Vogeley \& Dr. Gordon T. Richards \\
 & \\
\end{tabular*}
\end{tabbing}

\item Graduate Teaching Assistant ({\it Sept. 2008 - March 2014}) \\
\begin{tabbing}
xxxx\=xxxxxxxx\=xxxxxxxx\=xxxxxxxx\=\kill
\>\begin{tabular*}{0.9\linewidth}{l@{\extracolsep{\fill}}r}
Drexel Univ., Dept. of Physics & \\
Supervisors: Dr. Michel Vallieres & \\
Courses Taught: {\it Quantum Mechanics I, II, \& III} & \\ % PHYS 326, 327, \& 428
\ \ \ \ \ \ \ \ \ \ \ \ \ \ \ \ \ \ \ \ \ \ \ {\it Fundamentals of Physics I \& II} & \\ % PHYS 101, 102
\ \ \ \ \ \ \ \ \ \ \ \ \ \ \ \ \ \ \ \ \ \ \ {\it Introductory Physics I} & \\ % PHYS 152
 & \\
\end{tabular*}
\end{tabbing}

\item Adjunct Instructor ({\it June 2007 - June 2008}) \\
\begin{tabbing}
xxxx\=xxxxxxxx\=xxxxxxxx\=xxxxxxxx\=\kill
\>\begin{tabular*}{0.9\linewidth}{l@{\extracolsep{\fill}}r}
Virginia Commonwealth Univ., Dept. of Physics & \\
Supervisor: Dr. Alison A. Baski & \\
Courses Taught: {\it Elementary Astronomy} & \\ % PHYS 103
\ \ \ \ \ \ \ \ \ \ \ \ \ \ \ \ \ \ \ \ \ \ \ {\it General Physics I \& II} & \\ % PHYS 201, 202
\ \ \ \ \ \ \ \ \ \ \ \ \ \ \ \ \ \ \ \ \ \ \ {\it University Physics I \& II} & \\ % PHYS 207, 208
\ \ \ \ \ \ \ \ \ \ \ \ \ \ \ \ \ \ \ \ \ \ \ {\it Guided Inquiry for University Physics I \& II} & \\ % PHYS 351, 352
 & \\
\end{tabular*}
\end{tabbing}

\item Graduate Teaching Assistant ({\it Aug. 2005 - May 2007}) \\
\begin{tabbing}
xxxx\=xxxxxxxx\=xxxxxxxx\=xxxxxxxx\=\kill
\>\begin{tabular*}{0.9\linewidth}{l@{\extracolsep{\fill}}r}
Virginia Commonwealth Univ., Dept. of Physics & \\
Supervisor: Dr. Alison A. Baski & \\
Courses Taught: {\it Elementary Astronomy} & \\ % PHYS 103
\ \ \ \ \ \ \ \ \ \ \ \ \ \ \ \ \ \ \ \ \ \ \ {\it General Physics I \& II} & \\ % PHYS 201, 202
\ \ \ \ \ \ \ \ \ \ \ \ \ \ \ \ \ \ \ \ \ \ \ {\it University Physics I \& II} & \\ % PHYS 207, 208
 & \\
\end{tabular*}
\end{tabbing}

\item Graduate Research Assistant ({\it Summer 2006}) \\
\begin{tabbing}
xxxx\=xxxxxxxx\=xxxxxxxx\=xxxxxxxx\=\kill
\>\begin{tabular*}{0.9\linewidth}{l@{\extracolsep{\fill}}r}
Virginia Commonwealth Univ., Dept. of Physics & \\
Advisor: Dr. Alison A. Baski \\
 & \\
\end{tabular*}
\end{tabbing}

\item Research Assistant ({\it May 2003 - May 2005}) \\
\begin{tabbing}
xxxx\=xxxxxxxx\=xxxxxxxx\=xxxxxxxx\=\kill
\>\begin{tabular*}{0.9\linewidth}{l@{\extracolsep{\fill}}r}
%Univ. of Richmond, Dept. of Physics & \\
%Advisor: Dr. Emory F. Bunn \\
% & \\
%\end{tabular*}
%\end{tabbing}
\item Computing Lab Assistant ({\it Jan. 2002 - May 2005}) \\
%\begin{tabbing}
%xxxx\=xxxxxxxx\=xxxxxxxx\=xxxxxxxx\=\kill
%\>\begin{tabular*}{0.9\linewidth}{l@{\extracolsep{\fill}}r}
%Univ. of Richmond, Information Services & \\
%Supervisor: Vicki F. Brady \\
% & \\
%\end{tabular*}
%\end{tabbing}

\end{itemize}


\end{comment}


\newpage
\pagestyle{fancy}
\lhead{\textcolor{black}{\it Vishal Kasliwal}}
\fancyfoot[C]{}

\subsection*{REFERENCES}
\hrule  height 4pt
\vspace{0.2cm}

%{\it Upon request}

%\begin{comment}


\begin{itemize}
	\item Dr. Michael S. Vogeley \\
	Director of Graduate Studies; Professor
	\begin{tabbing}
	\=xxxxxxxx\=xxxxxxxx\=xxxxxxxx\=\kill
	\begin{tabular*}{\linewidth}{l@{\extracolsep{\fill}}r}

	Dept. of Physics & Phone: (215)895-2710 \\
	Drexel Univ. &  Email: vogeley@drexel.edu \\
	3141 Chestnut Street & \\
	Philadelphia, PA 19104 & \\
	\end{tabular*}
	\end{tabbing}

	\item Dr. Gordon T. Richards \\
        	Associate Professor
	\begin{tabbing}
	\=xxxxxxxx\=xxxxxxxx\=xxxxxxxx\=\kill
	\begin{tabular*}{\linewidth}{l@{\extracolsep{\fill}}r}

	Dept. of Physics & Phone: (215)895-2713 \\
	Drexel Univ. &  Email: gtr@physics.drexel.edu \\
	3141 Chestnut Street & \\
	Philadelphia, PA 19104 & \\
	\end{tabular*}
	\end{tabbing}

        \item Dr. Adam Lidz \\
        Associate Professor
	\begin{tabbing}
	\=xxxxxxxx\=xxxxxxxx\=xxxxxxxx\=\kill
	\begin{tabular*}{\linewidth}{l@{\extracolsep{\fill}}r}

	Dept. of Physics \& Astronomy & Phone: (215) 898-9597 \\
    Univ. of Pennsylvania &  Email: alidz@sas.upenn.edu \\
	203 S. 33rd St., & \\
	Philadelphia, PA 19104 & \\
	\end{tabular*}
	\end{tabbing}

        \item Dr. John D. Swinbank \\
        Professional Specialist
    \begin{tabbing}
    \=xxxxxxxx\=xxxxxxxx\=xxxxxxxx\=\kill
    \begin{tabular*}{\linewidth}{l@{\extracolsep{\fill}}r}

    Dept. of Astrophysical Sciences & Phone: (609) 258-3801 \\
    Princeton Univ. &  Email: jds@astro.princeton.edu \\
    4 Ivy Lane, & \\
    Princeton, NJ 08544 & \\
    \end{tabular*}
    \end{tabbing}

    \begin{comment}


        \item Dr. Robert Gilmore \\
        	Professor
	\begin{tabbing}
	\=xxxxxxxx\=xxxxxxxx\=xxxxxxxx\=\kill
	\begin{tabular*}{\linewidth}{l@{\extracolsep{\fill}}r}

	Dept. of Physics & Phone: (215)895-2779 \\
	Drexel Univ. &  Email: robert.gilmore@drexel.edu \\
	3141 Chestnut Street & \\
	Philadelphia, PA 19104 & \\
	\end{tabular*}
	\end{tabbing}


    \end{comment}


\end{itemize}
\vspace{0.1cm}

\end{document}

