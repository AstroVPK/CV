% Start a document with the here given default font size and paper size.
\documentclass[10pt,a4paper]{article}

% Set the page margins.
\usepackage[a4paper,margin=0.75in]{geometry}

% Setup the language.
\usepackage[english]{babel}
\hyphenation{Some-long-word}

% Makes resume-specific commands available.
\usepackage{resume}

\usepackage{verbatim}


\begin{document}  % begin the content of the document
\sloppy  % this to relax whitespacing in favour of straight margins


% title on top of the document
\maintitle{Vishal Pramod Kasliwal}{}{Last update on \today}

\nobreakvspace{0.3em}  % add some page break averse vertical spacing

% \noindent prevents paragraph's first lines from indenting
% \mbox is used to obfuscate the email address
% \sbull is a spaced bullet
% \href well..
% \\ breaks the line into a new paragraph
\noindent\href{mailto:vishal.dot.kasliwal.at.gmail.dot.com}{vishal.kasliwal\mbox{}@\mbox{}gmail.com}\sbull
\noindent\href{mailto:vishal.at.wavecomp.dot.com}{vishal\mbox{}@\mbox{}wavecomp.com}\sbull
\textsmaller{+}1.267.206.9287\sbull
%{\newnums cies010} \emph{(Skype)}\sbull
\href{https://github.com/AstroVPK}{https://github.com/AstroVPK}
\\
1065 Valencia Avenue.\sbull
\thinspace Apt. 1\sbull%{\large \sc apt 207}\sbull
Sunnyvale, CA 94086\sbull
USA

\spacedhrule{0.9em}{-0.4em}  % a horizontal line with some vertical spacing before and after


\begin{comment}


\roottitle{Summary}  % a root section title

\vspace{-1.3em}  % some vertical spacing
\begin{multicols}{2}  % open a multicolumn environment
\noindent \emph{Imaginative former astrophysicist with a passion for building numerical software that hides high-performance computing behind a friendly UI.}
\\
\\
Having fallen in love with Astronomy at the age of four, I was spurred by my family to follow my passion for astrophysics, mathematics, \& computing.

Coming from a {\large \sc gw basic} background in high-school, I learned {\large \sc idl} to search for non-Gaussianity in the Cosmic Microwave Background for my senior research project at the University of Richmond (2005). My {\large \sc idl} skills found application developing an automated image-analysis pipeline for studying the oxygen etching of silicon surfaces while obtaining my Master's degree at Virginia Commonwealth University (2007).

After beginning my doctorate at Drexel University, I used a variety of techniques, including machine learning with neural nets, to quantify the accuracy with which the Large Synoptic Survey Telescope (LSST) will measure galactic distances. The sheer volume of data generated by NASA's {\it Kepler} mission led to me programming Intel's Xeon Phi accelerator cards to study the variability of Active Galactic Nuclei for my PhD dissertation (2015).

My first postdoc with the \href{http://dm.lsst.org/}{LSST project} exposed me to the challenges of developing a high-performance software stack (15 TB/night over 10 years of raw data, \CPP \ \& Python, 50+ developers) in a professional Agile software development environment with an emphasis on algorithmic correctness \& coding standards. At the same time, working with \href{http://www.physics.upenn.edu/people/standing-faculty/adam-lidz}{Adam Lidz} at Upenn, I extended my \CPP \& Python package \href{https://github.com/AstroVPK/kali}{\textsc{k\={a}l\={i}}} to analyzing periodically-beamed stochastic light curves generated by dual black-hole systems in distant galaxies.

Bitten by the high-performance computing (HPC) bug, I recently began a research engineer position at a Si valley company, Colfax International, that provides HPC software consulting to a broad swath of clients from industry.
\end{multicols}

\spacedhrule{0em}{-0.4em}


\end{comment}


\roottitle{Experience}

\headedsection  % sets the header for the section and includes any subsections
  {\href{https://wavecomp.ai/}{Wave Computing - Research \& Development Software Engineer}}
  {\textsc{Campbell, CA}} {%

  \headedsubsection
    {Deep Learning Computational Kernel Development}
    {Dec \apo17 -- present}
    {\bodytext{Wave Computing is developing Dataflow Processing Units (DPUs), which contain thousands of interconnected dataflow Processing Elements (PEs). DPUs power Wave Computing's custom appliance for developing, testing, and deploying Deep Learning models. I have implemented computational kerels for IEEE 754-2008 rounding, ReLU, ReLU backpropagation, and bias backpropagation.}}
}

\headedsection  % sets the header for the section and includes any subsections
  {\href{https://colfaxresearch.com/}{Colfax International - High Performance Computing (HPC) Research Engineer}}
  {\textsc{Sunnyvale, CA}} {%

  \headedsubsection
    {HPC Computational Fluid Dynamics (CFD) Application Development}
    {Mar \apo17 -- Dec \apo17}
    {\bodytext{HPC consulting project. I parallelized a CFD simulation code written in C for a client in the oil \& gas sector resulting in a $\sim$ 10X speedup. I have also refactored the client's codebase to enable independent time-evolution in different regions of the simulation via domain-decomposition methods.}}

  \headedsubsection
    {C/C++ Compiler Analysis}
    {Nov \apo17}
    {\bodytext{HPC research project. I investigated suitability of C++ compilers for HPC applications. I developed optimized scientific computational kernels and investigated the  performance obtained by each compiler from each kernel. I analyzed compiled binary code to determine reason for differences in performance. A technical report of my findings can be obtained at \href{https://colfaxresearch.com/compiler-comparison/}{Colfax Research}.}}

  \headedsubsection
    {Intel Advisor Lecture}
    {May \apo17 -- Jul \apo17}
    {\bodytext{I presented a lecture on Intel Advisor for the Stanford University course ME 344: Introduction to High Performance Computing on July 20th, 2017.}}

}

\headedsection  % sets the header for the section and includes any subsections
  {\href{http://dm.lsst.org/}{Large Synoptic Survey Telescope (LSST) Data Management (Princeton University)}}
  {\textsc{Princeton, NJ}} {%

  \headedsubsection
    {Postdoctoral Research Associate}
    {Sept \apo15 -- Feb \apo17}
    {\bodytext{LSST Data Management is building a \CPP \ \& Python software stack to analyze raw imaging data from LSST.  I worked on the software stack to add functionality, documentation, \& tests. I developed \& implemented algorithm to propagate covariance when stacking images and worked on techniques for optimal image stacking \& differential chromatic refraction. I worked on a machine-learning based star-galaxy classifier and on converting the LSST stack to use {\sffamily py.test}. }}
}

\headedsection  % sets the header for a subsection and contains usually body text
  {\href{http://www.physics.upenn.edu/}{Department of Physics \& Astronomy (University of Pennsylvania)}}
  {\textsc{Philadelphia, PA}} {%

  \headedsubsection
    {Postdoctoral Researcher}
    {Sep \apo15 -- Feb \apo17}
    {\bodytext{I developed and implemented a parallelized Bayesian algorithm to estimate orbital parameters from stochastic light curves of binary supermassive black holes. I also developed and implemented Python framework to automatically wrangle astronomical time-series data from a variety of sources including web-servers, SQL servers, data servers and local data files.}}

  \headedsubsection
    {Principle Developer}
    {Sept \apo15 -- Feb \apo17}
    {\bodytext{I architected and implemented \href{https://github.com/AstroVPK/kali}{\textsc{k\={a}l\={i}}}, an open-source high performance library to model stochastic time-series data in a Bayesian framework. \href{https://github.com/AstroVPK/kali}{\textsc{k\={a}l\={i}}} is capable of modeling time-series data as variants of C-ARMA processes (a type of Gaussian random process). Written primarily in \CPP and exposed to Python using Cython, \href{https://github.com/AstroVPK/kali}{\textsc{k\={a}l\={i}}} uses {\sffamily scikit-learn} for machine learning, Intel MKL for fast linear algebra, Intel Bull Mountain technology for hardware random number generation, \& OpenMP 4.0 for vectorization \& parallelization. \href{https://github.com/AstroVPK/kali}{\textsc{k\={a}l\={i}}} is being used to study astronomical time-series data by multiple research groups at Caltech, UPenn, \& Drexel.}}
}

\headedsection
  {\href{http://drexel.edu/coas/academics/departments-centers/physics/}{Department of Physics (Drexel University)}}
  {\textsc{Philadelphia, PA}} {%

  \headedsubsection
    {AGN Variability Analysis}
    {June \apo09 -- Aug \apo15}
    {\bodytext{I developed \CPP sofwtare for Intel Xeon Phi accelerator cards to model AGN variability. I developed vectorized \& parallelized the \CPP \ pipeline to forward-model and fit data to model using MLE of $2^{\textrm{nd}}$-order statistics.}}

  \headedsubsection
    {LSST Photo-z Analysis}
    {Sept \apo08 -- May \apo09}
    {\bodytext{I used MLE \& machine learning (neural networks) to establish optimal y-band filter for LSST galaxy photo-z distance estimation.}}
}

\headedsection
  {\href{https://physics.vcu.edu/}{Department of Physics (Virginia Commonwealth University)}}
  {\textsc{Richmond, VA}} {%

  \headedsubsection
    {Adjunct Instructor}
    {Jun \apo07 -- Aug \apo08}
    {\bodytext{I taught {\it Introduction to Astronomy} course.}}

  \headedsubsection
    {AFM Image Analysis}
    {Aug \apo05 -- May \apo07}
    {\bodytext{I implemented an {\large \sc idl} pipeline to analyze AFM images of silicon surfaces etched using oxygen.}}
}

\headedsection  % sets the header for a subsection and contains usually body text
  {\href{http://physics.richmond.edu/}{Department of Physics (University of Richmond)}}
  {\textsc{Richmond, VA}} {%

  \headedsubsection
    {Cosmic Microwave Background Analysis}
    {May \apo03 -- May \apo05}
    {\bodytext{I used {\large \sc idl} to perform statistical tests of the utility of the bispectrum for detection of non-Gaussianity in the CMB.}}
}


\spacedhrule{0.4em}{-0.4em}


\roottitle{Education}

\headedsection
  {\href{http://drexel.edu/}{Drexel University}}
  {\textsc{Philadelphia, PA}} {%
  \headedsubsection
    {PhD. in Physics}
    {2008 -- 2015}
    {\bodytext{Probing AGN Accretion Physics through AGN Variability: Insights from {\it Kepler}}}
}

\headedsection
  {\href{http://www.vcu.edu/}{Virginia Commonwealth University}}
  {\textsc{Richmond, VA}} {%
  \headedsubsection
    {M.S in Physics \& Applied Physics}
    {2005 -- 2007}
    {\bodytext{CAFM Studies of Epitaxial Lateral Overgrowth GaN Films}}
}

\headedsection
  {\href{http://www.richmond.edu/}{University of Richmond}}
  {\textsc{Richmond, VA}} {%
  \headedsubsection
    {B.S. in Mathematics \& Physics}
    {2001 -- 2005}
    {\bodytext{The Bispectrum as a Quantifier of non-Gaussianity in the Cosmic Microwave Background}}
}


\spacedhrule{1.0em}{-0.4em}


\roottitle{Certifications}

  \headedsection
    {\href{https://www.deeplearning.ai/}{deeplearning.ai}}
    {\href{https://www.coursera.org/}{\textsc{coursera.org}}} {

      \headedsubsection
        {\href{https://www.coursera.org/account/accomplishments/verify/JSDREA8QF6XG}{Neural Networks and Deep Learning}}
        {\bodytext{Certificate earned on November 4, 2017.}}

      \headedsubsection
        {\href{https://www.coursera.org/account/accomplishments/verify/8FCH622WF97A}{Improving Deep Neural Networks: Hyperparameter tuning, Regularization and Optimization}}
        {\bodytext{Certificate earned on November 26, 2017.}}

      \headedsubsection
        {\href{https://www.coursera.org/account/accomplishments/verify/UXZBYZQKAEU9}{Structuring Machine Learning Projects}}
        {\bodytext{Certificate earned on December 17, 2017.}}

      \headedsubsection
        {\href{https://www.coursera.org/account/accomplishments/records/GNUR9KQXTNPS}{Convolutional Neural Networks}}
        {\bodytext{Certificate earned on January 22, 2018.}}

    }


\spacedhrule{1.0em}{-0.4em}

\roottitle{Skills}

  \inlineheadsection  % special section that has an inline header with a 'hanging' paragraph
  {Technical expertise:}
  {My expertise lies in the design \& implementation of high-perforamnce scientific/numerical software. I am highly skilled at applying statistical analysis and machine learning methodology to complex data and am familiar with developing Deep Learning applications using TensorFlow \& Keras. Experienced with working with(in) teams, I am fond of Agile methodologies (Scrum) and continuous integration (Jenkins). I enjoy writing C\nsp, \CPP\nsp, \& Python, and am learning Julia/\nsp Go/\nsp Io/\nsp Prolog. I have excellent knowledge of parallelization technologies:\ OpenMP, \CPP11 threads, POSIX threads, \& MPI as well as experience programming on novel computing platforms such Wave Computing's Dataflow Processing Units (DPU) \& Intel's Xeon Phi (Knight's Corner \& Knight's Landing). I have extensive knowledge \& experience with various programming toolchains including the GNU toolchain, Intel toolchain, LLVM toolchain, PGI toolchain, AOCC toolchain, Valgrind, gdb, make, SCons etc.... My preferred platform is Linux (16 years of development experience) although I have also developed on Mac OSX (7 years). I am very comfortable with writing in \LaTeX \ (13 years of experience) and have a good understanding of the UNIX programming environment and tools (Memory management, process spawning, etc...)}

  \vspace{1.0em}

  \inlineheadsection
    {Public speaking:}
    {With years of experience delivering highly technical talks to both expert \& general audiences, I am comfortable with public speaking \& outreach.}

  \vspace{1.0em}

  \inlineheadsection
    {Natural languages:}
    {English \emph{(native language)} and Hindi \emph{(native language)}.}

\spacedhrule{1.6em}{-0.8em}


\roottitle{Publications}

\inlineheadsection
  {A Performance-Based Comparison of C/C++ Compilers}{\href{https://colfaxresearch.com/compiler-comparison/}{Colfax Research}, 2017}

\inlineheadsection
  {Large Synoptic Survey Telescope Galaxies Science Roadmap}{\href{https://arxiv.org/abs/1708.01617}{arXiV}, 2017}

\inlineheadsection
  {Extracting Information from AGN Variability}{\href{https://doi.org/10.1093/mnras/stx1420}{MNRAS, 470, 3, 3027-3048}, 2017}

\inlineheadsection
  {Science-driven Optimization of the LSST Observing Strategy}{\href{https://github.com/LSSTScienceCollaborations/ObservingStrategy}{GitHub Repository}, 2016}

\inlineheadsection
  {The LSST Data Management System}{\href{http://adsabs.harvard.edu/cgi-bin/bib_query?arXiv:1512.07914}{Proceedings of ADASS XXV}, 2015}

\inlineheadsection
  {Do the Kepler AGN light curves need reprocessing?}{\href{http://dx.doi.org/10.1093/mnras/stv1797}{MNRAS, 453, 2075}, 2015}

\inlineheadsection
  {Are the variability properties of the Kepler AGN light curves consistent with a damped random walk?}{\href{http://dx.doi.org/ 10.1093/mnras/stv1230}{MNRAS, 451, 4328}, 2015}

\inlineheadsection
 {Thirty Meter Telescope Detailed Science Case: 2015}{\href{http://arxiv.org/abs/1505.01195}{http://arxiv.org/abs/1505.01195}, 2015}

\inlineheadsection
  {AFM and CAFM studies of ELO GaN films}{\href{http://dx.doi.org/10.1117/12.706773}{Proc. SPIE 6473, 647308}, 2007}

\inlineheadsection
  {Local electronic and optical behaviors of a-plane GaN grown via epitaxial lateral overgrowrth}{\href{http://dx.doi.org/10.1063/1.2429901}{Appl. Phys. Lett., 90, 011913}, 2007}


\spacedhrule{1.0em}{-0.4em}


\roottitle{Presentations}

\inlineheadsection
  {Intel Advisor}{ME344: Introduction to High Performance Computing, July 20th, 2017, Stanford University, CA}

\inlineheadsection
  {Optical Variability Signatures from Massive Black Hole Binaries}{229$^{\mathrm{th}}$ Meeting of the American Astronomical Society, 2017, Grapevine, TX}

\inlineheadsection
  {Extracting Information From AGN Variability: an LSST AGN Collaboration Proposal}{2017 LSST AGN Science Collaboration Roadmap Development Meeting, 2017, Grapevine, TX}

\inlineheadsection
  {Extracting Information from AGN Variability}{2016 KARL LSST Workshop, November 2016, Louisville, KY.}

\inlineheadsection
  {Surveying the Dynamic Sky with the LSST}{2016 KARL LSST Workshop, November 2016, Louisville, KY.}

\inlineheadsection
  {AGN Variability: Insights from Kepler}{2016 Hotwiring the Transient Universe V Meeting, October 2016, Villanova, PA.}

\inlineheadsection
  {Probing Accretion Processes through Variability}{2016 TMT Science Forum `International Partnership for Global Astronomy', May 2016, Kyoto, Japan.}

\inlineheadsection
  {AGN Variability: Insights from Kepler}{Princeton HSC Science Discussion Series, March 2016, Princeton, NJ.}

\inlineheadsection
  {AGN Variability on Short Timescales: What does Kepler tell us about AGN Variability?}{2015 TMT Science Forum `Maximizing Transformative Science with TMT', June 2015, Washington, DC.}

\inlineheadsection
  {What can Kepler tell us about AGN variability?}{225th Meeting of the American Astronomical Society, January 2015, Seattle, WA.}

\inlineheadsection
  {Do Kepler AGN Light Curves Exhibit a Damped Random Walk?}{24th Meeting of the American Astronomical Society, June 2014, Boston, MA.}

\inlineheadsection
  {The Bispectrum of Galactic Dust: Implications for Microwave Background non-Gaussianity}{204th Meeting of the American Astronomical Society, May 2004, Denver, CO.}


\end{document}
