% Start a document with the here given default font size and paper size.
\documentclass[10pt,a4paper]{article}

% Set the page margins.
\usepackage[a4paper,margin=0.75in]{geometry}

% Setup the language.
\usepackage[english]{babel}
\hyphenation{Some-long-word}

% Makes resume-specific commands available.
\usepackage{resume}




\begin{document}  % begin the content of the document
\sloppy  % this to relax whitespacing in favour of straight margins


% title on top of the document
\maintitle{Vishal Pramod Kasliwal}{}{Last update on \today}

\nobreakvspace{0.3em}  % add some page break averse vertical spacing

% \noindent prevents paragraph's first lines from indenting
% \mbox is used to obfuscate the email address
% \sbull is a spaced bullet
% \href well..
% \\ breaks the line into a new paragraph
\noindent\href{mailto:vishal.dot.kasliwal.at.gmail.dot.com}{vishal.kasliwal\mbox{}@\mbox{}gmail.com}\sbull
\textsmaller{+}1.267.206.9287\sbull
%{\newnums cies010} \emph{(Skype)}\sbull
\href{https://github.com/AstroVPK}{https://github.com/AstroVPK}
\\
2400 Chestnut St.\sbull
\thinspace Apt. 207\sbull%{\large \sc apt 207}\sbull
Philadelphia, PA\sbull
USA

\spacedhrule{0.9em}{-0.4em}  % a horizontal line with some vertical spacing before and after

\roottitle{Summary}  % a root section title

\vspace{-1.3em}  % some vertical spacing
\begin{multicols}{2}  % open a multicolumn environment
\noindent \emph{Imaginative astrophysicist with a passion for building scientific software that hides high-performance computing behind a friendly UI.}
\\
\\
Having fallen in love with Astronomy at the age of four, I was spurred by my family to follow my passion for astrophysics, mathematics, \& computing.

Coming from a {\large \sc gw basic} background in high-school, I learned {\large \sc idl} to search for non-Gaussianity in the Cosmic Microwave Background for my senior research project at the University of Richmond (2005). My {\large \sc idl} skills found application developing an automated image-analysis pipeline for studying the oxygen etching of silicon surfaces while obtaining my Master's degree at Virginia Commonwealth University (2007).

After beginning my doctorate at Drexel University, I used a variety of techniques, including machine learning with neural nets, to quantify the accuracy with which the Large Synoptic Survey Telescope (LSST) will measure galactic distances. The sheer volume of data generated by NASA's {\it Kepler} mission led to me programming Intel's Xeon Phi accelerator cards to study the variability of Active Galactic Nuclei for my PhD dissertation (2015).

My first postdoc with the LSST project has exposed me to the challenges of developing a high-performance software stack  (15 TB/night over 10 years of raw data, \CPP \ \& Python, 50+ developers) in a professional Agile software development environment with an emphasis on algorithmic correctness \& coding standards.
\end{multicols}


\spacedhrule{0em}{-0.4em}

\roottitle{Experience}

\headedsection  % sets the header for the section and includes any subsections
  {\href{http://dm.lsst.org/}{Large Synoptic Survey Telescope Data Management (Princeton University)}}
  {\textsc{Princeton, NJ}} {%

  \headedsubsection
    {Postdoctoral Research Associate}
    {Sept \apo15 -- present}
    {\bodytext{LSST DM is building the \CPP \ \& Python software stack to analyze the raw imaging data from the Large Synoptic Survey Telescope. Worked on the software stack to add functionality, documentation, \& tests. Converted the stack to use {\sffamily py.test}. Implemented covariance propagation when warping images. Researched (and currently implementing) solutions for optimal image stacking \& differential chromatic refraction.}}
}

\headedsection  % sets the header for a subsection and contains usually body text
  {\href{http://www.physics.upenn.edu/}{Department of Physics \& Astronomy (University of Pennsylvania)}}
  {\textsc{Philadelphia, PA}} {%

  \headedsubsection
    {Postdoctoral Researcher}
    {Sep \apo15 -- present}
    {\bodytext{Developed and implemented parallelized Bayesian algorithm to estimate orbital parameters from stochastic light curves of binary supermassive black holes. Developed and implemented Python framework to automatically wrangle astronomical time-series data from a variety of sources including web-servers, SQL servers, data servers and local data files.}}

  \headedsubsection
    {Principle Developer}
    {Sept \apo15 -- present}
    {\bodytext{Architected and largely implemented \textsc{k\={a}l\={i}}, an open-source high performance library to model stochastic time-series data in a Bayesian framework. \textsc{k\={a}l\={i}} is capable of modeling time-series data as variants of C-ARMA processes (a type of Gaussian random process). Written primarily in \CPP and exposed to Python using Cython, \textsc{k\={a}l\={i}} uses {\sffamily scikit-learn} for machine learning, Intel MKL for fast linear algebra, Intel Bull Mountain technology for hardware random number generation, \& OpenMP 4.0 for vectorization \& parallelization. \textsc{k\={a}l\={i}} is being used to study astronomical time-series data by multiple research groups at Caltech, UPenn, \& Drexel.}}
}

\headedsection
  {\href{http://drexel.edu/coas/academics/departments-centers/physics/}{Department of Physics (Drexel University)}}
  {\textsc{Philadelphia, PA}} {%

  \headedsubsection
    {AGN Variability Analysis}
    {June \apo09 -- Aug \apo15}
    {\bodytext{Used Intel Xeon Phi accelerator cards to model AGN variability. Developed vectorized \& parallelized \CPP \ pipeline to forward-model and fit data to model using MLE of $2^{\textrm{nd}}$-order statistics.}}

  \headedsubsection
    {LSST Photo-z Analysis}
    {Sept \apo08 -- May \apo09}
    {\bodytext{Used MLE \& machine learning (neural networks) to establish optimal y-band filter for LSST galaxy photo-z distance estimation.}}
}

\headedsection
  {\href{https://physics.vcu.edu/}{Department of Physics (Virginia Commonwealth University)}}
  {\textsc{Richmond, VA}} {%

  \headedsubsection
    {Adjunct Instructor}
    {Jun \apo07 -- Aug \apo08}
    {\bodytext{Taught {\it Introduction to Astronomy} course.}}

  \headedsubsection
    {AFM Image Analysis}
    {Aug \apo05 -- May \apo07}
    {\bodytext{Implemented {\large \sc idl} pipeline to analyze AFM images of silicon surfaces etched using oxygen.}}
}

\headedsection  % sets the header for a subsection and contains usually body text
  {\href{http://physics.richmond.edu/}{Department of Physics (University of Richmond)}}
  {\textsc{Richmond, VA}} {%

  \headedsubsection
    {Cosmic Microwave Background Analysis}
    {May \apo03 -- May \apo05}
    {\bodytext{Used {\large \sc idl} to perform statistical tests of the utility of the bispectrum for detection of non-Gaussianity in the CMB.}}
}


\spacedhrule{0.4em}{-0.4em}


\roottitle{Education}

\headedsection
  {\href{http://drexel.edu/}{Drexel University}}
  {\textsc{Philadelphia, PA}} {%
  \headedsubsection
    {PhD. in Physics}
    {2008 -- 2015}
    {\bodytext{Probing AGN Accretion Physics through AGN Variability: Insights from {\it Kepler}}}
}

\headedsection
  {\href{http://www.vcu.edu/}{Virginia Commonwealth University}}
  {\textsc{Richmond, VA}} {%
  \headedsubsection
    {M.S in Physics \& Applied Physics}
    {2005 -- 2007}
    {\bodytext{CAFM Studies of Epitaxial Lateral Overgrowth GaN Films}}
}

\headedsection
  {\href{http://www.richmond.edu/}{University of Richmond}}
  {\textsc{Richmond, VA}} {%
  \headedsubsection
    {B.S. in Mathematics \& Physics}
    {2001 -- 2005}
    {\bodytext{The Bispectrum as a Quantifier of non-Gaussianity in the Cosmic Microwave Background}}
}


\spacedhrule{0.5em}{-0.4em}

\roottitle{Skills}

\inlineheadsection  % special section that has an inline header with a 'hanging' paragraph
  {Technical expertise:}
  {Scientific software design and implementation, with(in) a team.  Fond of Agile methodologies (Scrum) and continuous integration (Jenkins).  Enjoys writing Python/\nsp \CPP, and learning Julia/\nsp Go/\nsp Io/\nsp Prolog.  Good knowledge of parallelization technologies:\ OpenMP, Cilk+, \CPP11 threads, POSIX threads, \& MPI.  Good knowledge of programming tools:\ GNU toolchain, Intel tools, Valgrind, gdb, make, SCons etc.... Development experience on Linux (16 years) \& Mac OSX (7 years). Experience with \LaTeX \ (13 years). Good knowledge of UNIX programming environment and tools:\ Memory management, process spawning, etc... Familiar with MS Windows.}

\vspace{0.5em}
\inlineheadsection
  {Natural languages:}
  {English \emph{(native language)} and Hindi \emph{(native language)}.}


\spacedhrule{1.6em}{-0.4em}


\roottitle{Publications}

\inlineheadsection
  {Science-driven Optimization of the LSST Observing Strategy}{\href{https://github.com/LSSTScienceCollaborations/ObservingStrategy}{GitHub Repository}, 2016}

\inlineheadsection
  {Extracting Information from AGN Variability}{\href{https://arxiv.org/abs/1607.04299}{\textit{submitted to MNRAS}}, 2016}

\inlineheadsection
  {The LSST Data Management System}{\href{http://adsabs.harvard.edu/cgi-bin/bib_query?arXiv:1512.07914}{Proceedings of ADASS XXV}, 2015}

\inlineheadsection
  {Do the Kepler AGN light curves need reprocessing?}{\href{http://dx.doi.org/10.1093/mnras/stv1797}{MNRAS, 453, 2075}, 2015}

\inlineheadsection
  {Are the variability properties of the Kepler AGN light curves consistent with a damped random walk?}{\href{http://dx.doi.org/ 10.1093/mnras/stv1230}{MNRAS, 451, 4328}, 2015}

\inlineheadsection
 {Thirty Meter Telescope Detailed Science Case: 2015}{\href{http://arxiv.org/abs/1505.01195}{http://arxiv.org/abs/1505.01195}, 2015}

\inlineheadsection
  {AFM and CAFM studies of ELO GaN films}{\href{http://dx.doi.org/10.1117/12.706773}{Proc. SPIE 6473, 647308}, 2007}

\inlineheadsection
  {Local electronic and optical behaviors of a-plane GaN grown via epitaxial lateral overgrowrth}{\href{http://dx.doi.org/10.1063/1.2429901}{Appl. Phys. Lett., 90, 011913}, 2007}


\spacedhrule{1.0em}{-0.4em}


\roottitle{Presentations}

\inlineheadsection
  {Extracting Information from AGN Variability}{2016 KARL LSST Workshop, November 2016, Louisville, KY.}

\inlineheadsection
  {Surveying the Dynamic Sky with the LSST}{2016 KARL LSST Workshop, November 2016, Louisville, KY.}

\inlineheadsection
  {AGN Variability: Insights from Kepler}{2016 Hotwiring the Transient Universe V Meeting, October 2016, Villanova, PA.}

\inlineheadsection
  {Probing Accretion Processes through Variability}{2016 TMT Science Forum `International Partnership for Global Astronomy', May 2016, Kyoto, Japan.}

\inlineheadsection
  {AGN Variability: Insights from Kepler}{Princeton HSC Science Discussion Series, March 2016, Princeton, NJ.}

\inlineheadsection
  {AGN Variability on Short Timescales: What does Kepler tell us about AGN Variability?}{2015 TMT Science Forum `Maximizing Transformative Science with TMT', June 2015, Washington, DC.}

\inlineheadsection
  {What can Kepler tell us about AGN variability?}{225th Meeting of the American Astronomical Society, January 2015, Seattle, WA.}

\inlineheadsection
  {Do Kepler AGN Light Curves Exhibit a Damped Random Walk?}{24th Meeting of the American Astronomical Society, June 2014, Boston, MA.}

\inlineheadsection
  {The Bispectrum of Galactic Dust: Implications for Microwave Background non-Gaussianity}{204th Meeting of the American Astronomical Society, May 2004, Denver, CO.}


\end{document}
